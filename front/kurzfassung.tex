\chapter{\textbf{Kurzfassung}}

Representational State Transfer (REST)-APIs sind heute weit verbreitet und aufgrund ihrer Einfachheit, Skalierbarkeit und Flexibilität werden sie überwiegend als Standardprotokoll für die Web-APIs angesehen. Es ist anzunehmen, dass die Bedeutung von desktopbasierten Anwendungen kontinuierlich abnimmt und immer mehr Benutzer von Desktop- zu Web- und weiteren mobilen Anwendungen wechseln.

Bei der Entwicklung von REST-basierten Web-Anwendungen wird ein REST-basierter Web Service benötigt, um die Funktionalitäten der Web-Anwendung richtig testen zu können. Da die gängigen Penetrationstest-Werkzeuge für REST-APIs nicht direkt einsatzfähig sind, wird die Sicherheit solcher APIs jedoch immer noch zu selten überprüft und das Testen dieser Arten von Anwendungen ist eine große Herausforderung. Grundsätzlich ist das erstmalige Testen für den Betreiber von Webanwendungen unüberschaubar. Verschiedene Werkzeuge, Frameworks und Bibliotheken dienen dazu da, die Testaktivität automatisieren zu können. Die Nutzer wählen diese Dienstprogramme basierend auf ihrem Kontext, ihrer Umgebung, ihrem Budget und ihrem Qualifikationsniveau. Einige Eigenschaften von REST-APIs machen es jedoch für automatisierte Web-Sicherheitsscanner schwierig, geeignete REST-API-Sicherheitstests für die Schwachstellen durchzuführen.

Diese Bachelorarbeit untersucht, wie Sicherheitstests heute durchgeführt werden und ermittelt qualitativ-deskriptiv, ob solche Sicherheitstests zuverlässig sind. Es werden verschiedene Methoden verglichen, die das Testen von Webanwendungen unterstützen. Dann werden ihre Vor- und Nachteile erarbeitet und gegeneinander abgewägt. Es werdem zudem die jeweiligen Schwachstellen und Angriffspunkte von Webanwendungen dargelegt.
Darüber hinaus wird eine Spring Boot Anwendung entwickelt, die eine REST- Schnittstelle mit Sicherheitslücken beinhaltet und ein Penetrationstest mit dem Open API 2.0 Plugin von OWASP Zap\footnote{https://www.owasp.org} evaluiert.

Im Rahmen dieser Bachelorarbeit wird außerdem ein Wegweiser für die Entwicklung des Open API 3.0 Plugins für das Open Source Werkzeug OWASP Zap erstellt, indem die Unterschiede zwischen Open API 2.0 (Swagger)\footnote{https://swagger.io/specification/v2/} und Open API 3.0\footnote{https://swagger.io/specification/} gezeigt werden. Des Weiteren wird erfasst, welche Mängel in API 2.0 bestehen, welche Unterschiede es zu Open API 3.0 gibt und ob es notwendig ist, eben dieses Plugin zu entwickeln. Schlussendlich soll diese Arbeit herausfinden, inwiefern REST-Dokumente auch bei automatisierten Penetrationstests relevant sind und genutzt werden können.
