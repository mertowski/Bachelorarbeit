\chapter{Kurzfassung}

REST-APIs sind heutzutage weit verbreitet und dank ihrer Einfachheit, Skalierbarkeit und Flexibilität werden sie weitgehend als Standardprotokoll für die Web-APIs angesehen. Es scheint plausibel zu sein anzunehmen, dass die Ära der Desktop-basierten Anwendungen kontinuierlich zurückgeht und im Zuge dessen, die Benutzer von Desktop- zu Web- und weiteren mobilen Anwendungen wechseln.

Bei der Entwicklung von REST-basierten Web-Anwendungen wird ein REST-basierter Web Service benötigt, um die Funktionalitäten der Web-Anwendung richtig testen zu können. Da die gängigen Penetrationstest-Werkzeuge für REST-APIs nicht direkt einsatzfähig sind, wird die Sicherheit solcher APIs jedoch immer noch zu selten überprüft und das Testen dieser Arten von Anwendungen ist eine sehr große Herausforderung. Grundsätzlich ist das erstmalige Testen für den Betreiber von Webanwendungen sehr unüberschaubar. Verschiedene Werkzeuge, Frameworks und Bibliotheken sind dazu da, die Testaktivität automatisieren zu können. Die Nutzer wählen diese Dienstprogramme basierend auf ihrem Kontext, ihrer Umgebung, ihrem Budget und ihrem Qualifikationsniveau. Einige Eigenschaften von REST-APIs machen es jedoch für automatisierte Web-Sicherheitsscanner schwierig, geeignete REST-API-Sicherheitstests für die Schwachstellen durchzuführen.

Diese Bachelorarbeit untersucht wie die Sicherheitstests heutzutage realisiert werden und versucht qualitativ-deskriptiv aufzudecken, ob auf solche Sicherheitstests Verlass ist. Es werden verschiedene Methoden verglichen, die das Testen von RESTful APIs unterstützen. Dann wird ihre Vor- und Nachteile herausgefunden und gegeneinander abgewägt. Es wird auch Gewissheit verschaffen, wie die jeweiligen Schwachstellen und Angriffspunkte von Webanwendungen dargelegt. Es wird noch eine Spring Boot- Anwendung mit Sicherheitslücken entwickelt und wird eine Penetrationstest mit dem Open API 2.0 Plugin von OWASP Zap evaluiert.

Im Rahmen dieser Bachelorarbeit wird außerdem ein Wegweiser für die Entwicklung des Open API 3.0 Plugins für das Open Source Werkzeug OWASP Zap erstellt, indem die Unterschiede zwischen Open API 2.0 und Open API 3.0 gezeigt werden. Des Weiteren wird versucht zu erfassen, was genau in API 2.0 fehlt, welche Unterschiede sich zu Open API 3.0 zeigen und ob überhaupt eine Notwendigkeit besteht, eben dieses Plugin zu entwickeln. Schlussendlich strebt diese Arbeit an herauszufinden, ob REST-Dokumente bei einem Penetrationstest eine Rolle spielen und wie groß diese Rolle bei einem Penetrationstest wäre.