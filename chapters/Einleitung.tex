\chapter{Einleitung}
\label{chap:k1}

Bei \LaTeX\ werden Absätze durch freie Zeilen angegeben.
Da die Arbeit über ein Versionskontrollsystem versioniert wird, ist es sinnvoll, pro \emph{Satz} neue Zeile im \texttt{.java}-Dokument anzufangen.
So kann einfacher ein Vergleich von Versionsständen vorgenommen werden.

Die Arbeit ist in folgender Weise gegliedert:
In \cref{chap:k2} werden die Grundlagen dieser Arbeit beschrieben.
Schließlich fasst \cref{chap:k2} die Ergebnisse der Arbeit zusammen und stellt Anknüpfungspunkte vor.

\section{Motivation}

bla bla..


\section{Zielsetzung}

bla bla..


\section{Aufbau der Arbeit}

\begin{enumerate}
\item Vorerst werden in Kapitel 2..
\item In einer Anforderungsanalyse in Kapitel 3..
\item Kapitel 4 vermittelt..
\item Es folgt irgendwas in Kapitel 5. In diesem Teil..
\item Kapitel 6 dokumentiert
\item Die Gesamtarbeit..
\end{enumerate}

