\chapter{Einleitung}
\label{cha:Einleitung}

\section{Hintergrund und Motivation}

Im Berufsalltag sind Webanwendungen zu einem unverzichtbaren Bestandteil vieler moderner Unternehmen geworden. Ein effektiv implementiertes System kann nicht nur einen reibungslosen Betrieb ermöglichen, sondern auch die Managementprozesse erheblich verbessern. Solche Systeme bringen bemerkenswerte Vorteile mit sich, aber auch verheerende Folgen und Verluste, wenn das System von Hackern übernommen wird. Daher sind verschiedene Abwehrmechanismen erforderlich, um Eindringlinge zu verhindern.

Sobald Sicherheitsmaßnahmen ergriffen wurden, stellt sich die Frage, wie effektiv sie tatsächlich sind. Hier sind Penetrationstests von Bedeutung. Als einer der gebräuchlichsten Ansätze zur Bewertung der Systemsicherheit sind Penetrationstests Simulationen von Aktionen, die von Hackern ausgeführt werden, um ein IT-System zu infiltrieren. Das Hauptziel von Penetrationstests besteht darin, potenzielle Sicherheitslücken im System zu identifizieren. Dieser Prozess ermöglicht es den Entwicklern, pragmatische Lösungen zu finden, um diese Schwachstellen zu beheben.

\section{Zielsetzung}

Diese Bachelorarbeit besitzt mehr als nur einen Schwerpunkt, dabei handelt es sich um folgende:

\begin{itemize}
	\item manuelle Penetrationstests für Schwachstellen,
	\item automatisierter Penetrationstest einer lokalen Webanwendung auf Schwachstellen,
	\item automatisierter REST-API-Penetrationstest einer Spring Boot Anwendung mit dem Plugin OpenAPI 2.0 von OWASP ZAP,
	\item Erstellen eines Wegweisers für das OpenAPI 3.0 Plugin, indem die Unterschiede zwischen beiden Frameworks gezeigt werden.\\
\end{itemize}

Sicherheitsrisiken, die bei Webanwendungen auftreten können, werden analysiert und detailliert erklärt. Anschließend wird gezeigt, wie manuelle und automatische Tests ablaufen und welche der bestehenden Testmethoden zum welchem Zeitpunkt sinnvoll sind. Im Anschluss wird ein automatischer Test einer Spring Boot Anwendung mit Hilfe des OpenAPI 2.0 Plugins durchgeführt und dabei versucht, die Bedeutung der REST-API bei der Entwicklung zu evaluieren. Abschließend werden alle Unterschiede zwischen Swagger (OpenAPI) 2.0 und OpenAPI 3.0 dargestellt. Dabei werden die Neuerungen von OpenAPI 3.0 sowie die Notwendigkeit von Änderungen des API 3.0 Plugins für OWASP ZAP erörtert.

\section{Aufbau der Arbeit}

Durch die Auseinandersetzung mit den Grundlagen in Kapitel 2 wird zunächst ein generelles Grundwissen zur IT-Sicherheit geschaffen.

Des Weiteren werden die Sicherheitsrisiken von Webanwendungen in Kapitel \ref{cha:k3} detailliert erläutert. 

In Kapitel \ref{cha:k4} wird ein Überblick über die Penetrationstests gegeben, indem Informationen über die Ziele, die Kriterien und den Ablauf eines Penetrationstests vermittelt werden. Außerdem werden manuelle und automatisierte Penetrationstests gezeigt. Es wird erarbeitet, welche Vor- und Nachteile diese zwei Formen des Penetrationstests aufweisen.

Es folgt die Evaluierung des OpenAPI 2.0 Plugins von OWASP ZAP für die REST-API einer Spring Boot Anwendung.

In Kapitel 6 werden Open API 2.0 und Open API 3.0 miteinander verglichen und daraufhin bewertet.

Mit dem Kapitel 7 wird die Bachelorthesis abgeschlossen, indem Vorschläge für Penetrationstests formuliert werden.


