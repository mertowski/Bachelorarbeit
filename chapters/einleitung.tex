\chapter{Einleitung}
\label{cha:Einleitung}

\section{Hintergrund und Motivation}

\begin{quote}
	\begin{center}
		\emph{\\
			"`Knowing your enemy is the key to winning the battle."'}
	\end{center}
	\begin{flushright}
		- Sun Tzu
	\end{flushright}
\end{quote}

In der heutigen Geschäftswelt sind Webanwendungen zu einem untrennbaren Bestandteil vieler moderner Unternehmen geworden. Ein effektiv implementiertes System kann nicht nur einen reibungslosen Betrieb ermöglichen, sondern auch die Managementprozesse erheblich verbessern. Solche Systeme bringen sich bemerkenswerten Vorteile, aber auch sie können verheerende Folgen und Verluste erleiden, wenn das System von Hackern übernommen wird. Daher sind im Wesentlichen verschiedenste Abwehrmechanismen erforderlich, um Eindringlinge zu verhindern.

Sobald die Sicherheitsmaßnahmen ergriffen sind, stellt sich schließlich die Frage, wie effektiv sie tatsächlich sind. Hier spielen die Penetrationstests sehr große Rolle. Als einer der gebräuchlichsten Ansätze zur Bewertung der Systemsicherheit können Penetrationstests als die Simulation von Aktionen betrachtet werden, die von Hackern ausgeführt werden, um ein IT-System zu infiltrieren, aber das Hauptziel von Penetrationstests besteht darin, potenzielle Sicherheitslücken im System zu identifizieren. Dieser Prozess ermöglicht es den Penetrationstester pragmatische Lösungen zu finden, um diese Schwachstellen zu beheben.

\section{Zielsetzung}

Durch die Bachelorarbeit soll folgendes aufgezeigt werden;

\begin{itemize}
	\item Manuelle Penetrationstests für Schwachstellen,
	\item Automatisiertes Penetrationstest einer lokalen Webanwendung auf Schwachstellen,
	\item Automatisiertes REST API Penetrationstest einer Springboot Anwendung mit dem Plug-In OpenAPI 2.0 von OWASP ZAP,
	\item Erstellen einem Wegweiser für das OpenAPI 3.0 Plug-In, indem die Unterschiede zwischen beiden Frameworks zu zeigen.\\
\end{itemize}

Ziel des Penetrationstests war es, Schwachstellen der Webanwendungen abzubilden und diese Schwachstellen möglichst auszunutzen. Darüber hinaus demonstrieren die Testergebnisse die Bedeutung der Aufrechterhaltung sicherer Systeme.

Ein zweites Ziel des Penetrationstests bestand darin, möglicherweise die Sicherheitslücken bei der Springboot-Anwendung finden und Verbesserungen vorschlagen.

Ein zweites Ziel des Penetrationstests bestand darin, möglicherweise die Sicherheitslücken bei der Springboot-Anwendung finden und Verbesserungen vorschlagen.

Ein zweites Ziel des Penetrationstests bestand darin, möglicherweise die Sicherheitslücken bei der Springboot-Anwendung finden und Verbesserungen vorschlagen.

\section{Aufbau der Arbeit}

Vorerst wird eine gemeinsame Basiswissen der IT-Sicherheit geschaffen, indem in Kapitel 2 die Grundlagen behandelt werden. 

Des Weiter werden die Sicherheitsrisiken von Webanwendungen in Kapitel 3  detailliert erläutert. Kapitel 4 vermittelt einen Überblick über die Penetrationstests, indem Informationen über die Ziele, Kriterien und Ablauf eines Penetrationstest geschaffen werden. Außerdem werden die manuelle und automatisierte Penetrationstests angezeigt und versucht raus zu finden, welche Vor- und Nachteile zwischen manuelle und automatisierte Penetrationstests haben. 

Es folgt die Evaluierung von OpenAPI 2.0 Plug-In von OWASP ZAP für die REST API einer Springboot Anwendung. 

Kapitel 6 dokumentiert die Vergleich und Bewertung zwischen Open API 2.0 und Open API 3.0. 

Am Ende schließt Kapitel 7 die Bachelorthesis ab und schlägt eine Reihe von Vorschlägen für Penetrationstest vor.

