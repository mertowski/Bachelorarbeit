\chapter{Fazit}
\label{cha:k7}

%%%%%%%%%%%%%%%%%%AUTO ODER MANUAL%%%%%%%%%%%%%%%%%%%%

As mentioned earlier, sometimes it is not possible to cover the ins and outs of the target system or perform fuzzing manually using large number of payloads. In such cases, we can use an automated tool to do our job. It saves a lot of manual effort and time.
Automated tools can also be used for information gathering techniques, which can be very useful before starting the discovery phase. Hence, in such cases, we can use an automated tool to find the right target after which we can use manual assessment to exploit the vulnerability. Even in cases where the size of the application is large, an automated security scan comes handy. However, the result given by the automated tool isn’t necessarily the conclusion. A manual analysis is often required to confirm the vulnerabilities. Manual techniques are also helpful in finding business logic flaws. Thus, a mix of both automated and manual testing would be the best fit to save time and get the best output.



Automated penetration testing tools tend to be more efficient and thorough, and chances are that malicious hackers are going to use automated attacks against you. These automated test tools come from many sources, including commercial, open-source and custom designed. Often these tools focus on a particular vulnerability area, so multiple penetration testing tools may be needed. Because these automated tools are updated monthly or weekly, you must manually verify the output from the automated tools to check for false alarms and to test for the latest vulnerabilities. With over 50 new vulnerabilities being discovered each week, there will always be new vulnerabilities that the tools may not be able to detect. Without doing this manual testing, your penetration testing will be incomplete. 

Zusammenfassend soll nicht in die Debatte zwischen manuellen und automatisierten Tests verwickelt werden. Beide dienen ihren individuellen Zwecken wunderbar. Es soll das optimale Gleichgewicht bei dem Testen von jede Anwendung gefunden werden.

%%%%%%%%%%%%%%%%%%%%%%%%%%%%%%%%%%%%%%%%%%%%%%%%%





%%%%%%%%%%%%%%%%%%SWAGGER VS OPENAPI%%%%%%%%%%%%%%%%%%%%

Swagger 2.0 ist aufgrund der hohen Verbreitung und der umfangreichen Toolunterstützung der Quasistandard für die Schnittstellenbeschreibung REST-basierter Anwendungen. Die OpenAPI Specification 3.0.0 sorgt für Ordnung in der gewachsenen Swagger-Struktur. Durch ein breites Gremium, in dem alle namhaften Hersteller vertreten sind, und die neuen Features, wird sich die OpenAPI Specification in der Branche durchsetzen und Swagger 2.0 ablösen, sobald die Toolunterstützung auf die neue Version angepasst ist.

%%%%%%%%%%%%%%%%%%%%%%%%%%%%%%%%%%%%%%%%%%%%%%%%%%%%