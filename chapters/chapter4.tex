\chapter{Penetrationstest}
\label{chap:k4}

Durch der Penetrationstests kann geprüft werden, inwiefern die Sicherheit der IT-Systeme durch Bedrohungen von Hackern, Crackern, etc. gefährdet ist bzw. ob die IT-Sicherheit durch die eingesetzten Sicherheitsmaßnahmen aktuell gewährleistet ist. 

\section{Grundlegendes Konzept}

\subsection{Black-Box}

\subsection{White-Box}

\subsection{Gray-Box}

\section{Kriterien für Penetrationstests}

\section{Ablauf eines Penetrationstest}

\section{Penetrationstest-Tools}

\subsection{ABC}

\section{Automatisierte Penetrationstest-Tools}

\subsection{ABC}

\section{Vor- und Nachteile zwischen manuelle und automatisierte Penetrationstest-Tools}