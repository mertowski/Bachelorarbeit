\chapter{Penetrationstest}
\label{chap:k2}

Hier wird der Hauptteil stehen. Falls mehrere Kapitel gewünscht, entweder mehrmals \texttt{\textbackslash{}chapter} benutzen oder pro Kapitel eine eigene Datei anlegen und \texttt{ausarbeitung.tex} anpassen.

LaTeX-Hinweise stehen in \cref{chap:latextipps}.

\section{Grundlegendes Konzept}

\subsection{Black-Box}

\subsection{White-Box}

\subsection{Gray-Box}

\section{Kriterien für Penetrationstests}

\section{Ablauf eines Penetrationstest}

\section{Rechtliche Aspekte}

\section{Sicherheitstest-Tools}

\subsection{ABC}

\section{Automatisierte Sicherheitstest-Tools}

\subsection{ABC}

\section{Vor- und Nachteile zwischen manuelle und automatisierte Sicherheitstest-Tools}