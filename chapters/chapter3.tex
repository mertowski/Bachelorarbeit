\chapter{Sicherheitsrisiken von Webanwendungen}
\label{chap:k3}

In diesem Kapitel wird die Sicherheit von Webanwendungen anhand von Bedrohungen, Gegenmaßnahmen, Schwachstellen und Angriffen analysiert.

\section{Schwachstellen}

\subsection{OWASP Top 10 Risiken}

\subsubsection{Injection}

Injektion-Schwachstellen wie SQL-, NoSQL-, OS- und LDAP-Injection treten auf, wenn nicht vertrauenswürdige Daten als Teil eines Befehls oder einer Datenabfrage von einem Interpreter verarbeitet werden\cite[6]{owasp17top10}.

\subsubsection{Fehler in Authentifizierung und Session-Management}

asd

\subsubsection{Cross-Site Scripting (XSS)}

\subsubsection{Unsichere direkte Objektreferenzen}

\subsubsection{Sicherheitsrelevante Fehlkonfiguration}

\subsubsection{Verlust der Vertraulichkeit sensibler Daten}

\subsubsection{Fehlerhafte Autorisierung auf Anwendungsebene}

\subsubsection{Cross-Site Request Forgery (CSRF)}

\subsubsection{Nutzung von Komponenten mit bekannten Schwachstellen}

\subsubsection{Ungeprüfte Um- und Weiterleitungen}

\subsection{Weitere Risiken}

\subsection{Common Vulnerability Scoring System}

\subsection{Common Vulnerability Exposures (CVE)}

\section{Technische Schwachstellen}

\subsection{Programmierfehler}

\subsection{Konfigurationsfehler}

\subsection{Konzeptionsfehler}

\subsection{Fehler resultierend aus menschlichem Verhalten}