\chapter{Grundlagen}
\label{chap:k2}

Hier wird der Hauptteil stehen. Falls mehrere Kapitel gewünscht, entweder mehrmals \texttt{\textbackslash{}chapter} benutzen oder pro Kapitel eine eigene Datei anlegen und \texttt{ausarbeitung.tex} anpassen.

LaTeX-Hinweise stehen in \cref{chap:latextipps}.

\section{Informationssicherheit}

\section{Technologien des Projektes}

\section{Penetrationstest}

\subsection{Kriterien für Penetrationstests}

\subsection{Ablauf eines Penetrationstest}

\subsection{Rechtliche Aspekte}

\subsection{Sicherheitstest-Tools}

\subsection{Automatisierte Sicherheitstest-Tools}

\section{Sicherheitsrisiken von Webanwendungen}

\section{Schwachstellen}

\subsection{OWASP Top 10 Risiken}

\subsection{Weitere Risiken}

\subsection{Common Vulnerability Scoring System}

\subsection{Common Vulnerability Exposures (CVE)}

\section{Technische Schwachstellen}

\subsection{Programmierfehler}

\subsection{Konfigurationsfehler}

\subsection{Konzeptionsfehler}

\subsection{Fehler resultierend aus menschlichem Verhalten}