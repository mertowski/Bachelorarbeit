% !TeX spellcheck = de_DE
%Die Angabe des schlauen Spruchs auf diesem Wege funtioniert nur,
%wenn keine Änderung des Kapitels mittels den in preambel/chapterheads.tex
%vorgeschlagenen Möglichkeiten durchgeführt wurde.
\setchapterpreamble[u]{%
\dictum[Albert Einstein]{Probleme kann man niemals mit derselben Denkweise lösen, durch die sie entstanden sind.}
}
\chapter{Benutzerhandbuch für das OWASP-Zap RAML Plug-in}
\label{chap:guide}

Pro Satz eine neue Zeile.
Das ist wichtig, um sauber versionieren zu können.
In LaTeX werden Absätze durch eine Leerzeile getrennt.

Folglich werden neue Abstäze insbesondere \emph{nicht} durch Doppelbackslashes erzeugt.
Der letzte Satz kam in einem neuen Absatz.

\section{Quellcode}
\Cref{lst:ListingANDlstlisting} zeigt, wie man Programmlistings einbindet.
Mittels \texttt{\textbackslash lstinputlisting} kann man den Inhalt direkt aus Dateien lesen.

%Listing-Umgebung wurde durch \newfloat{Listing} definiert
\begin{Listing}
\begin{lstlisting}
<listing name="second sample">
  <content>not interesting</content>
</listing>
\end{lstlisting}
\caption{lstlisting in einer Listings-Umgebung, damit das Listing durch Balken abgetrennt ist}
\label{lst:ListingANDlstlisting}
\end{Listing}

Quellcode im \lstinline|<listing />| ist auch möglich.

\section{Abbildungen}

Die \cref{fig:chor1} und \ref{fig:chor2} sind für das Verständnis dieses Dokuments wichtig.
Im Anhang zeigt \vref{fig:AnhangsChor} erneut die komplette Choreographie.

%Die Parameter in eckigen Klammern sind optionale Parameter - z.B. [htb!]
%htb! bedeutet: "Liebes LaTeX, bitte platziere diese Abbildung zuerst hier ("_h_ere"). Falls das nicht funktioniert, dann bitte oben auf der Seite ("_t_op"). Und falls das nicht geht, bitte unten auf der Seite ("_b_ottom"). Und bitte, bitte bevorzuge hier und oben, auch wenn's net so optimal aussieht ("!")
%Diese sollten nach Möglichkeit NICHT verwendet werden. LaTeX's Algorithmus für das Platzieren der Gleitumgebung ist schon sehr gut!
\begin{figure}
  \centering
  \includegraphics[width=\textwidth]{choreography.pdf}
  \caption{Beispiel-Choreographie}
  \label{fig:chor1}
\end{figure}

\begin{figure}
  \centering
  \includegraphics[width=.8\textwidth]{choreography.pdf}
  \caption[Beispiel-Choreographie]{Die Beispiel-Choreographie. Nun etwas kleiner, damit \texttt{\textbackslash textwidth} demonstriert wird. Und auch die Verwendung von alternativen Bildunterschriften für das Verzeichnis der Abbildungen. Letzteres ist allerdings nur Bedingt zu empfehlen, denn wer liest schon so viel Text unter einem Bild? Oder ist es einfach nur Stilsache?}
  \label{fig:chor2}
\end{figure}


\begin{figure}
  \centering
    \subfloat[]{\includegraphics[width=0.3\textwidth]{choreography.pdf} \label{fig:subfigA}}
    \subfloat[]{\includegraphics[width=0.3\textwidth]{choreography.pdf} \label{fig:subfigB}}
		\subfloat[Subcaption if needed]{\includegraphics[width=0.3\textwidth]{choreography.pdf} \label{fig:subfigC}}
	\caption{Beispiel um 3 Abbildung nebeneinader zu stellen nur jedes einzeln referenzieren zu können. Abbildung~\ref{fig:subfigB}
 ist die mittlere Abbildung.}
\label{fig:subfig_example}
\end{figure}

Es ist möglich, SVGs direkt beim Kompilieren in PDF umzuwandeln.
Dies ist im Quellcode zu latex-tipps.tex beschrieben, allerdings auskommentiert.

\iffalse % <-- Das hier wegnehmen, falls inkscape im Pfad ist
Das SVG in \cref{fig:directSVG} ist direkt eingebunden, während der Text im SVG in \cref{fig:latexSVG} mittels pdflatex gesetzt ist.
Falls man die Graphiken sehen möchte, muss inkscape im PATH sein und im Tex-Quelltext \texttt{\textbackslash{}iffalse} und \texttt{\textbackslash{}iftrue} auskommentiert sein.

\begin{figure}
\centering
\includegraphics{svgexample.svg}
\caption{SVG direkt eingebunden}
\label{fig:directSVG}
\end{figure}

\begin{figure}
\centering
\def\svgwidth{.4\textwidth}
\includesvg{svgexample}
\caption{Text im SVG mittels \LaTeX{} gesetzt}
\label{fig:latexSVG}
\end{figure}
\fi % <-- Das hier wegnehmen, falls inkscape im Pfad ist

\section{Tabellen}

\cref{tab:Ergebnisse} zeigt Ergebnisse und die \cref{tab:Ergebnisse} zeigt wie numerische Daten in einer Tabelle representiert werden können.
\begin{table}
  \centering
  \begin{tabular}{ccc}
  \toprule
  \multicolumn{2}{c}{\textbf{zusammengefasst}} & \textbf{Titel} \\ \midrule
  Tabelle & wie & in \\
  \url{tabsatz.pdf}& empfohlen & gesetzt\\

  \multirow{2}{*}{Beispiel} & \multicolumn{2}{c}{ein schönes Beispiel}\\
   & \multicolumn{2}{c}{für die Verwendung von \enquote{multirow}}\\
  \bottomrule
  \end{tabular}
  \caption[Beispieltabelle]{Beispieltabelle -- siehe \url{http://www.ctan.org/tex-archive/info/german/tabsatz/}}
  \label{tab:Ergebnisse}
\end{table}

\begin{table}
	\centering
	\begin{tabular}{l *{8}{d{3.2}}}
		\toprule
						
			   & \multicolumn{2}{c}{\textbf{Parameter 1}} & \multicolumn{2}{c}{\textbf{Parameter 2}} & \multicolumn{2}{c}{\textbf{Parameter 3}} & \multicolumn{2}{c}{\textbf{Parameter 4}} \\
			\cmidrule(r){2-3}\cmidrule(lr){4-5}\cmidrule(lr){6-7}\cmidrule(l){8-9}
			
			\textbf{Bedingungen} & \multicolumn{1}{c}{\textbf{M}} & \multicolumn{1}{c}{\textbf{SD}} & \multicolumn{1}{c}{\textbf{M}} & \multicolumn{1}{c}{\textbf{SD}} & \multicolumn{1}{c}{\textbf{M}} & \multicolumn{1}{c}{\textbf{SD}} & \multicolumn{1}{c}{\textbf{M}} & \multicolumn{1}{c}{\textbf{SD}}\\
			\midrule
			
			W & 1.1 & 5.55 & 6.66 & .01 &  &  &  & \\
			X & 22.22 & 0.0 & 77.5 & .1 &  &  &  & \\
			Y & 333.3 & .1 & 11.11 & .05 &  &  &  & \\
			Z & 4444.44 & 77.77 & 14.06 & .3 &  &  &  & \\
		\bottomrule 
	\end{tabular}
	
	\caption{Beispieltabelle f\"{u}r 4 Bedingungen (W-Z) mit jeweils 4 Parameters mit (M und SD). Hinweiß: immer die selbe anzahl an Nachkommastellen angeben.}
	\label{tab:Werte}
\end{table}

\section{Abkürzungen}

Beim ersten Durchlauf betrug die \gls{fr} 5.
Beim zweiten Durchlauf war die \gls{fr} 3.~Die Pluralform sieht man hier:\ \glspl{er}.
Um zu demonstrieren, wie das Abkürzungsverzeichnis bei längeren Beschreibungstexten aussieht, muss hier noch \glspl{rdbms} erwähnt werden.

Mit \verb+\gls{...}+ können Abkürzungen eingebaut werden, beim ersten Aufrufen wird die lange Form eingesetzt.
Beim wiederholten Verwenden von \verb+\gls{...}+ wird automatisch die kurz Form angezeigt.
Außerdem wird die Abkürzung automatisch in die Abkürzungsliste eingefügt.
Mit \verb+\glspl{...}+ wird die Pluralform verwendet.
Möchte man, dass bei der ersten Verwendung direkt die Kurzform erscheint, so kann man mit \verb+\glsunset{...}+ eine Abkürzung als bereits verwendet markieren.
Das Gegenteil erreicht man mit \verb+\glsreset{...}+.

Definiert werden Abkürzungen in der Datei \textit{content\\ausarbeitung.tex} mithilfe von \verb+\newacronym{...}{...}{...}+.

Mehr Infos unter: \url{http://tug.ctan.org/macros/latex/contrib/glossaries/glossariesbegin.pdf}

\section{Definitionen}
\begin{definition}[Title]
\label{def:def1}
Definition Text
\end{definition}

\Cref{def:def1} zeigt \ldots

\section{Fußnoten}
Fußnoten können mit dem Befehl \verb+\footnote{...}+ gesetzt werden\footnote{\label{fussnote}Diese Fußnote ist ein Beispiel.}. Mehrfache Verwendung von Fußnoten ist möglich indem man zu erst ein Label in der Fußnote setzt \verb+\footnote{\label{...}...}+ und anschließend mittels \verb+\cref{...}+ die Fußnote erneut verwendet\cref{fussnote}.

\section{Verschiedenes}
\label{sec:diff}
\ifdeutsch
Ziffern (123\,654\,789) werden schön gesetzt.
Entweder in einer Linie oder als Minuskel-Ziffern.
Letzteres erreicht man durch den Parameter \texttt{osf} bei dem Paket \texttt{libertine} bzw.\ \texttt{mathpazo} in \texttt{fonts.tex}.
\fi

\textsc{Kapitälchen} werden schön gesperrt...

\begin{compactenum}[I.]
\item Man kann auch die Nummerierung dank paralist kompakt halten
\item und auf eine andere Nummerierung umstellen
\end{compactenum}

